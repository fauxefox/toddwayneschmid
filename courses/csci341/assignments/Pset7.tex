\documentclass[11pt]{article}

\newcommand{\pset}{
    7
}
\newcommand{\subtitle}{
    From Double Pumping to the \(\lambda\)-calculus
}
\newcommand{\duedate}{
    Friday, October 31
}

% Page Setup
\usepackage{geometry}
\geometry{
    a4paper,
    margin={2.5cm}
}

% Basic Packages
\usepackage{amssymb}
\usepackage{stmaryrd}
\usepackage{amsmath}
\usepackage{amsthm}
\usepackage{mathtools}
\usepackage{mathpartir}
\usepackage{enumitem}
\usepackage{mathabx}

% Font
\usepackage{charter}

% Bibliography and index
\usepackage[backend=biber, style=numeric]{biblatex}
\addbibresource{refs.bib}
\usepackage{makeidx}
\makeindex

% Colors and Graphics
\usepackage[dvipsnames, x11names]{xcolor}
\usepackage{tikz}
\usetikzlibrary{
    cd,
    fit,
    calc,
    positioning,
    arrows,
    automata,
    shapes
}
\tikzset{
    baseline = (current bounding box.center),
    every state/.append style = {
        rectangle,
        rounded corners=5pt,
		inner sep = 3pt,
		minimum size = 18pt,
		initial text = {},
        fill=Azure1
	},
	every edge/.append style = {
		->,
		>=stealth,
		bend angle=10,
		thick
	}
}
\usepackage{musicography}
\usepackage{graphicx}
\usepackage{svg}
\graphicspath{../imgs/}

% Hyperlinks
\usepackage{hyperref}
\hypersetup{
    colorlinks,
    linkcolor   = black,
    filecolor   = RubineRed,
    urlcolor    = RubineRed,
    citecolor   = RubineRed,
    pdftitle    = {CSCI 341 Course Materials}
}
\usepackage[capitalize]{cleveref}

% Environments
\theoremstyle{theorem} % In Italics
\newtheorem{theorem}                    {{\color{Purple}Theorem}}[section]
\newtheorem{lemma}          [theorem]   {{\color{Magenta}Lemma}}
\newtheorem{proposition}    [theorem]   {Proposition}
\newtheorem{corollary}      [theorem]   {Corollary}
\newtheorem{question}                   {{\color{red}Question}}

\theoremstyle{definition} % Not in italics
\newtheorem{definition}     [theorem]   {{\color{NavyBlue}Definition}}
\newtheorem{example}        [theorem]   {{\color{ForestGreen}Example}}
\newtheorem{problem}                    {{\color{BurntOrange}Problem}}

\theoremstyle{remark} % Subdued label
\newtheorem{remark}[theorem]        {{\color{Gray}Remark}}

% (1), (2), ...
\renewcommand\labelenumi{(\theenumi)}

% Go nuts with line breaks 
\allowdisplaybreaks

%%%%%%%%%%
% MACROS %
%%%%%%%%%%

\newcommand{\op}{\mathrm{op}}               % Opposite
\newcommand{\inv}{{-1}}                     % Inverse
\newcommand{\id}{\mathsf{id}}               % Identity f(x) = x
\newcommand{\Det}{\mathrm{Det}}             % determinize
\newcommand{\Lang}{\mathcal{L}}             % Language

\newcommand{\incl}{\mathsf{incl}}           % Inclusion
\newcommand{\proj}{\mathsf{proj}}           % Projection

% Numbers and Standard notation
\newcommand{\NN}{\mathbb{N}}                % 0, 1, 2, 3, 4, ...
\newcommand{\ZZ}{\mathbb{Z}}                % ..., -2, -1, 0, 1, 2, ...
\newcommand{\QQ}{\mathbb{Q}}                % n/m for n and m in \NN and m > 0
\newcommand{\RR}{\mathbb{R}}                % real numbers
\newcommand{\pRR}{\mathbb{R}_{+}}           % positive real numbers

\newcommand{\dom}{\mathrm{dom}}             % Domain
\newcommand{\cod}{\mathrm{cod}}             % Codomain

\newcommand{\Grph}{\operatorname{Grph}}     % Graph of a function

% Transitions
\newcommand{\tr}[1]{
    \mathrel{
        \raisebox{-1pt}{
            \(\xrightarrow{#1}\)
        }
    }
}
\newcommand{\bisim}{\mathrel{\raisebox{1pt}{\(\underline{\leftrightarrow}\)}}}

% Text
\newcommand{\code}[1]{\texttt{#1}}
\newcommand{\codeblock}[1]{
    \begin{center}
        \parbox{0.8\textwidth}{
            \ttfamily
            #1
        }
    \end{center}
}

% Boolean statements
\newcommand{\OR}{~\mathrm{or}~}
\newcommand{\AND}{~\mathrm{and}~}
\newcommand{\NOT}{\mathrm{not}~}
\newcommand{\IMPLIES}{~\mathrm{implies}~}
\newcommand{\FORALL}{\mathrm{for\ all}~}
\newcommand{\EXISTS}{\mathrm{there\ exists}~}
\newcommand{\SUCHTHAT}{~\mathrm{such\ that}~}



% Title
\title{CSCI 341 Problem Set \pset}
\author{\subtitle}
\date{Due
    \duedate
}

\begin{document}

\maketitle

Don't forget to check the webspace for hints and additional context for each problem!

\begin{problem}
    [Three's a Crowd]
    Prove that the following language in the alphabet \(0,1\) is not context-free:
    \[
        L = \{w 0 w 0 w \mid w \in \{0,1\}^*\}
    \]
\end{problem}

\begin{proof}[Solution.]
    
\end{proof}

\begin{problem}
    [5 out of 16]
    Consider the decision problem 
    \[
        D_5 = \{5n \mid n \in \mathbb N\} \subseteq \mathbb{N}
    \]
    Given a natural number \(n\in \mathbb N\), let \(\mathsf{hex}(n)\) be the hexidecimal representation of \(n\). 
    Find a regular expression \(r\) such that \((\mathsf{hex}, \mathcal L(r))\) is a faithful representation of \(D_5\).
\end{problem}

\begin{proof}[Solution.]
    
\end{proof}

\begin{problem}
    [Composing Representations]
    In this problem, we are going to show that representations of functions ``compose''.
    We need a bit of notation: given functions \(f_1 \colon S_1 \to S_2\) and \(f_2 \colon S_2 \to S_3\), we are going to define \(f_2 \circ f_1 \colon S_1 \to S_3\) to be the function defined by 
    \[
        f_2 \circ f_1(s) = f_2(f_1(s))
    \]
    This function, \(f_2 \circ f_1\), is called the <i>composition of \(f_1\) and \(f_2\)</i>.

    <p></p>
    Now on to the problem.
    Let \(S_1,S_2,S_3\) be sets and let \(A\) be an alphabet.
    Let \(\rho_i \colon S_i \to A^*\) be a string representation for each \(i = 1,2,3\), and let \(f_1 \colon S_1 \to S_2\) and \(f_2 \colon S_2 \to S_3\) be functions. 
    \[
        S_1 \xrightarrow{f_1} S_2 \xrightarrow{f_2} S_3
    \]
    Given a representation \((\rho_1, g_1, \rho_2)\) of \(f_1\) and a representation \((\rho_2, g_2, \rho_3)\) of \(f_2\), prove that \((\rho_1, g_2 \circ g_1, \rho_3)\) is a representation of \(f_2 \circ f_1\).
\end{problem}

\begin{proof}
    [Solution.]
\end{proof}

\begin{problem}
    [OR WHAT]
    Let \(\vee \colon B \times B \to B\) be the logical "or" function. 
    Find a \(\lambda\)-representation \(\mathsf{OR}\) of \(\vee\), and evaluate its truth table.
\end{problem}

\begin{proof}
    [Solution.]
\end{proof}

\begin{problem}
    [Multipy by Three]
    Find a \(\lambda\)-term \(\mathsf{M}_3\) that represents <i>multiplication by \(3\)</i>. 
    That is, if \(\#_{Ch} \colon \mathbb N \to \lambda\mathit{Term}\) is the Church-numeral representation of natural numbers, \[
        \mathsf{M}_3\mathsf{C_n} \Downarrow \mathsf{C_{3n}}
    \]
    Verify that \(\mathsf{M_3} \mathsf{C_3} = \mathsf{C_9}\) using your definition of \(\mathsf{M_3}\).
\end{problem}

\begin{proof}
    [Solution.]
\end{proof}

\end{document}