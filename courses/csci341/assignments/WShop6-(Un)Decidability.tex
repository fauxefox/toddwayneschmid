\documentclass[11pt]{article}

\newcommand{\wshop}{
    5
}
\newcommand{\subtitle}{
    Stack Automata
}

% Page Setup
\usepackage{geometry}
\geometry{
    a4paper,
    margin={2.5cm}
}

% Basic Packages
\usepackage{amssymb}
\usepackage{stmaryrd}
\usepackage{amsmath}
\usepackage{amsthm}
\usepackage{mathtools}
\usepackage{mathpartir}
\usepackage{enumitem}
\usepackage{mathabx}

% Font
\usepackage{charter}

% Bibliography and index
\usepackage[backend=biber, style=numeric]{biblatex}
\addbibresource{refs.bib}
\usepackage{makeidx}
\makeindex

% Colors and Graphics
\usepackage[dvipsnames, x11names]{xcolor}
\usepackage{tikz}
\usetikzlibrary{
    cd,
    fit,
    calc,
    positioning,
    arrows,
    automata,
    shapes
}
\tikzset{
    baseline = (current bounding box.center),
    every state/.append style = {
        rectangle,
        rounded corners=5pt,
		inner sep = 3pt,
		minimum size = 18pt,
		initial text = {},
        fill=Azure1
	},
	every edge/.append style = {
		->,
		>=stealth,
		bend angle=10,
		thick
	}
}
\usepackage{musicography}
\usepackage{graphicx}
\usepackage{svg}
\graphicspath{../imgs/}

% Hyperlinks
\usepackage{hyperref}
\hypersetup{
    colorlinks,
    linkcolor   = black,
    filecolor   = RubineRed,
    urlcolor    = RubineRed,
    citecolor   = RubineRed,
    pdftitle    = {Notes on Behavioural PDEs}
}
\usepackage[capitalize]{cleveref}

% Environments
\theoremstyle{theorem} % In Italics
\newtheorem{theorem}                    {{\color{Purple}Theorem}}[section]
\newtheorem{lemma}          [theorem]   {{\color{Magenta}Lemma}}
\newtheorem{proposition}    [theorem]   {Proposition}
\newtheorem{corollary}      [theorem]   {Corollary}
\newtheorem{question}                   {{\color{red}Question}}

\theoremstyle{definition} % Not in italics
\newtheorem{definition}     [theorem]   {{\color{NavyBlue}Definition}}
\newtheorem{example}        [theorem]   {{\color{ForestGreen}Example}}
\newtheorem{problem}                    {{\color{BurntOrange}Problem}}

\theoremstyle{remark} % Subdued label
\newtheorem{remark}[theorem]        {{\color{Gray}Remark}}

% (1), (2), ...
\renewcommand\labelenumi{(\theenumi)}

% Go nuts with line breaks 
\allowdisplaybreaks

%%%%%%%%%%
% MACROS %
%%%%%%%%%%

\newcommand{\op}{\mathrm{op}}               % Opposite
\newcommand{\inv}{{-1}}                     % Inverse
\newcommand{\id}{\mathsf{id}}               % Identity f(x) = x
\newcommand{\Det}{\mathrm{Det}}             % determinize
\newcommand{\Lang}{\mathcal{L}}             % Language

\newcommand{\incl}{\mathsf{incl}}           % Inclusion
\newcommand{\proj}{\mathsf{proj}}           % Projection

% Numbers and Standard notation
\newcommand{\NN}{\mathbb{N}}                % 0, 1, 2, 3, 4, ...
\newcommand{\ZZ}{\mathbb{Z}}                % ..., -2, -1, 0, 1, 2, ...
\newcommand{\QQ}{\mathbb{Q}}                % n/m for n and m in \NN and m > 0
\newcommand{\RR}{\mathbb{R}}                % real numbers
\newcommand{\pRR}{\mathbb{R}_{+}}           % positive real numbers

\newcommand{\dom}{\mathrm{dom}}             % Domain
\newcommand{\cod}{\mathrm{cod}}             % Codomain

\newcommand{\Grph}{\operatorname{Grph}}     % Graph of a function

% Transitions
\newcommand{\tr}[1]{
    \mathrel{
        \raisebox{-1pt}{
            \(\xrightarrow{#1}\)
        }
    }
}
\newcommand{\bisim}{\mathrel{\raisebox{1pt}{\(\underline{\leftrightarrow}\)}}}

% Text
\newcommand{\code}[1]{\texttt{#1}}
\newcommand{\codeblock}[1]{
    \begin{center}
        \parbox{0.8\textwidth}{
            \ttfamily
            #1
        }
    \end{center}
}

% Boolean statements
\newcommand{\OR}{~\mathrm{or}~}
\newcommand{\AND}{~\mathrm{and}~}
\newcommand{\NOT}{\mathrm{not}~}
\newcommand{\IMPLIES}{~\mathrm{implies}~}
\newcommand{\FORALL}{\mathrm{for\ all}~}
\newcommand{\EXISTS}{\mathrm{there\ exists}~}
\newcommand{\SUCHTHAT}{~\mathrm{such\ that}~}



% Title
\title{CSCI 341 Workshop \wshop}
\author{\subtitle}
% \date{
%     \today
% }

\pagestyle{empty}

\begin{document}


\maketitle

%%%%%%%%%%%%%%%%%%%%%%%%%%%%%%%%%%%%%%%%%%%%%%%%%%%%%%%%%%%%
% START OF WORK SHOP.                                      %
%%%%%%%%%%%%%%%%%%%%%%%%%%%%%%%%%%%%%%%%%%%%%%%%%%%%%%%%%%%%

\begin{problem}
    \label{firstprob}
    Show that the language \(L_{=} = \{0^n 1^n \mid n \in \mathbb N\}\) is decidable.
\end{problem}

% \pagebreak

\begin{problem}
    We are going to show that the language 
    \[
        L_{11?} = \{\lfloor \mathcal T \rfloor \mathtt{*}x \mid \text{\(x\) halts on input \(\mathtt{11}\) in the Turing machine \(\mathcal T\)}\}
    \]
    is undecidable as follows: 
    \begin{enumerate}
        \item Write a program \(\mathtt{rep\_11}\) that clears the tape and then prints \(\mathtt{11}\) to the tape (i.e., if \(\mathcal C\) is the Turing machine containing the state \(\mathtt{rep\_11}\), then \(\mathcal C_{\mathtt{rep\_11}}(w) = \mathtt{11}\) for any \(w \in A^*\)).
        
        \item As an example of what you'll do next: in Problem~\ref{firstprob}, you designed a Turing machine \(\mathcal T\) with a state \(x\) such that \(\mathcal T_x(0^n1^n) = 1\) and otherwise \(\mathcal T_x(w) = 0\).
        Write a program \(\mathtt{rep\_11\_equal}\) that (1) clears the tape, (2) writes \(\mathtt{11}\) to the tape, then (3) runs \(\mathcal T\) starting from \(x\).

        \item Now describe a Turing machine \(\mathcal W\) with a state \(y\) that incorporates the code for \(\mathtt{rep\_11}\) into the encoding of any other Turing program, in the following way:
        if \(\mathcal T\) is a Turing machine with a state \(x\), then 
        \(\mathcal W_y(\lfloor \mathcal T\rfloor \mathtt{*}x)\) is an encoding of the Turing program that (1) clears the tape, (2) writes \(\mathtt{11}\) to the tape, then (3) runs \(\mathcal T\) starting from \(x\).
        
        \item We are now in a position to reduce \(L_{Halt}\) to \(L_{11?}\).\\
        Suppose that the Turing program \(x_{11}\) in the the Turing machine \(\mathcal E\) is a decider for the language \(L_{11?}\).
        Now we define the Turing machine \(\mathcal H\) and its program \(x_{hlt}\) as follows: 
        given a Turing machine \(\mathcal T\) with state \(x\) and an input word \(w \in \{0,1\}^*\), running \(x_{hlt}\) starting with the input string \(\lfloor \mathcal T\rfloor \mathtt{*}x \mathtt{*} w\)
        \begin{enumerate}
            \item runs \(\mathcal W_y(\lfloor \mathcal T\rfloor \mathtt{*}x \mathtt{*} w)\) and writes its contents to the tape
        \end{enumerate}


    \end{enumerate}
\end{problem}

\pagebreak



\end{document}